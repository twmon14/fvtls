%% 
%% Copyright 2019-2021 Elsevier Ltd
%% 
%% This file is part of the 'CAS Bundle'.
%% --------------------------------------
%% 
%% It may be distributed under the conditions of the LaTeX Project Public
%% License, either version 1.2 of this license or (at your option) any
%% later version.  The latest version of this license is in
%%    http://www.latex-project.org/lppl.txt
%% and version 1.2 or later is part of all distributions of LaTeX
%% version 1999/12/01 or later.
%% 
%% The list of all files belonging to the 'CAS Bundle' is
%% given in the file `manifest.txt'.
%% 
%% Template article for cas-dc documentclass for 
%% double column output.

\documentclass[a4paper,fleqn]{cas-dc}

% If the frontmatter runs over more than one page
% use the longmktitle option.

%\documentclass[a4paper,fleqn,longmktitle]{cas-dc}

%\usepackage[numbers]{natbib}
%\usepackage[authoryear]{natbib}
\usepackage[authoryear,longnamesfirst]{natbib}

%%%Author macros
\def\tsc#1{\csdef{#1}{\textsc{\lowercase{#1}}\xspace}}
\tsc{WGM}
\tsc{QE}
%%%

% Uncomment and use as if needed
%\newtheorem{theorem}{Theorem}
%\newtheorem{lemma}[theorem]{Lemma}
%\newdefinition{rmk}{Remark}
%\newproof{pf}{Proof}
%\newproof{pot}{Proof of Theorem \ref{thm}}

\begin{document}
\let\WriteBookmarks\relax
\def\floatpagepagefraction{1}
\def\textpagefraction{.001}

% Short title
\shorttitle{<short title of the paper for running head>}    

% Short author
\shortauthors{<short author list for running head>}  

% Main title of the paper
\title [mode = title]{Formal verification of TLS with CiMPG}  

% Title footnote mark
% eg: \tnotemark[1]
\tnotemark[1] 

% Title footnote 1.
% eg: \tnotetext[1]{Title footnote text}
\tnotetext[<tnote number>]{<tnote text>} 

% First author
%
% Options: Use if required
% eg: \author[1,3]{Author Name}[type=editor,
%       style=chinese,
%       auid=000,
%       bioid=1,
%       prefix=Sir,
%       orcid=0000-0000-0000-0000,
%       facebook=<facebook id>,
%       twitter=<twitter id>,
%       linkedin=<linkedin id>,
%       gplus=<gplus id>]

% \author[<aff no>]{<author name>}[<options>]
\author{}
% Corresponding author indication
% \cormark[<corr mark no>]

% Footnote of the first author
% \fnmark[<footnote mark no>]

% Email id of the first author
\ead{<email address>}

% URL of the first author
\ead[url]{<URL>}

% Credit authorship
% eg: \credit{Conceptualization of this study, Methodology, Software}
\credit{<Credit authorship details>}

% Address/affiliation
\affiliation[<aff no>]{organization={},
            addressline={}, 
            city={},
%          citysep={}, % Uncomment if no comma needed between city and postcode
            postcode={}, 
            state={},
            country={}}

% \author[<aff no>]{<author name>}[<options>]

% Footnote of the second author
% \fnmark[2]

% Email id of the second author
\ead{}

% URL of the second author
\ead[url]{}

% Credit authorship
\credit{}

% Address/affiliation
\affiliation[<aff no>]{organization={},
            addressline={}, 
            city={},
%          citysep={}, % Uncomment if no comma needed between city and postcode
            postcode={}, 
            state={},
            country={}}

% Corresponding author text
\cortext[1]{Corresponding author}

% Footnote text
\fntext[1]{}

% For a title note without a number/mark
%\nonumnote{}

% Here goes the abstract
\begin{abstract}
TLS 1.0 has been formally verified with CiMPG in CafeInMaude. The formal specification of TLS is modeled as transition systems in terms of equations in CafeOBJ/OTS method, and it has been verified that the system have properties by means of equational reasoning. The given initial proof scores in CafeOBJ are revised and added some additional proof fragments in some invariants if it is necessary such that the proof scores are complete and they could be successfully tackled by CiMPG. When the complete and correct proof scores are available, they are verified in CiMPG. Feeding the proof scores into CiMPG, CiMPG can generate proof scripts for CiMPA. However, it is not a good idea to feed all proof scores of all invariants in CiMPG as one file because CiMPG takes too much time to generate the proof scripts since the size of the proof score is quite large. Thus, each invariant has been tackled with CiMPG one by one.
\end{abstract}

% Use if graphical abstract is present
%\begin{graphicalabstract}
%\includegraphics{}
%\end{graphicalabstract}

% Research highlights
% \begin{highlights}
% \item 
% \item 
% \item 
% \end{highlights}

% Keywords
% Each keyword is seperated by \sep
\begin{keywords}
 CiMPG \ CafeInMaude \ invariant \ OTS 
\end{keywords}

\maketitle

% Main text
\section{Introduction}\label{intro}
Internet security has become the important topic in recent years. The rapid development of e-commerce, individuals and organizations are relying on the Web and security services such as online banking, online shops that need to exchange the sensitive information such as credit card numbers, account numbers, etc. between two entities in non-secure network environment. With the increase service availability the number of risks increase both for users and providers. In order to ensure a secure communication between them, a large number of security protocols have been designed and developed to protect the data. Among them, Transport Layer Security protocol (TLS)\cite{dierk} is the most widely used security protocol. TLS is the successor of Secure
Sockets Layer (SSL). 

% Numbered list
% Use the style of numbering in square brackets.
% If nothing is used, default style will be taken.
%\begin{enumerate}[a)]
%\item 
%\item 
%\item 
%\end{enumerate}  

% Unnumbered list
%\begin{itemize}
%\item 
%\item 
%\item 
%\end{itemize}  

% Description list
%\begin{description}
%\item[]
%\item[] 
%\item[] 
%\end{description}  

% Figure
% \begin{figure}[<options>]
% 	\centering
% 		\includegraphics[<options>]{}
% 	  \caption{}\label{fig1}
% \end{figure}


% \begin{table}[<options>]
% \caption{}\label{tbl1}
% \begin{tabular*}{\tblwidth}{@{}LL@{}}
% \toprule
%   &  \\ % Table header row
% \midrule
%  & \\
%  & \\
%  & \\
%  & \\
% \bottomrule
% \end{tabular*}
% \end{table}

% Uncomment and use as the case may be
%\begin{theorem} 
%\end{theorem}

% Uncomment and use as the case may be
%\begin{lemma} 
%\end{lemma}

%% The Appendices part is started with the command \appendix;
%% appendix sections are then done as normal sections
%% \appendix
% There are several ideas/devices that should be used to make the formal
% verification successful.
% You should describe TLS 1.0, CafeInMaude, CiMPA, CiMPG, how to
% formally specify TLS 1.0 and how to write proof scores.
\section{Preliminaries}\label{pre}

\section{How to formpally specify TLS 1.0} \label{fs}
\section{How to write proof scores}\label{proofscores}
\section{Formal verification of TLS with CiMPG}\label{fvtls}
\section{Related Work}\label{relatedwork}
\section{Conclusion}\label{conclusion}

% To print the credit authorship contribution details
\printcredits

%% Loading bibliography style file
%\bibliographystyle{model1-num-names}
\bibliographystyle{cas-model2-names}

% Loading bibliography database
\bibliography{cas-refs}

% Biography
\bio{}
% Here goes the biography details.
\endbio

% \bio{pic1}
% % Here goes the biography details.
% \endbio

\end{document}

% 