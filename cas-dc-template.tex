%% 
%% Copyright 2019-2021 Elsevier Ltd
%% 
%% This file is part of the 'CAS Bundle'.
%% --------------------------------------
%% 
%% It may be distributed under the conditions of the LaTeX Project Public
%% License, either version 1.2 of this license or (at your option) any
%% later version.  The latest version of this license is in
%%    http://www.latex-project.org/lppl.txt
%% and version 1.2 or later is part of all distributions of LaTeX
%% version 1999/12/01 or later.
%% 
%% The list of all files belonging to the 'CAS Bundle' is
%% given in the file `manifest.txt'.
%% 
%% Template article for cas-dc documentclass for 
%% double column output.

\documentclass[a4paper,fleqn]{cas-dc}

% If the frontmatter runs over more than one page
% use the longmktitle option.

%\documentclass[a4paper,fleqn,longmktitle]{cas-dc}

%\usepackage[numbers]{natbib}
%\usepackage[authoryear]{natbib}
\usepackage[authoryear,longnamesfirst]{natbib}

%%%Author macros
\def\tsc#1{\csdef{#1}{\textsc{\lowercase{#1}}\xspace}}
\tsc{WGM}
\tsc{QE}
%%%

% Uncomment and use as if needed
%\newtheorem{theorem}{Theorem}
%\newtheorem{lemma}[theorem]{Lemma}
%\newdefinition{rmk}{Remark}
%\newproof{pf}{Proof}
%\newproof{pot}{Proof of Theorem \ref{thm}}

\begin{document}
\let\WriteBookmarks\relax
\def\floatpagepagefraction{1}
\def\textpagefraction{.001}

% Short title
\shorttitle{<short title of the paper for running head>}    

% Short author
\shortauthors{<short author list for running head>}  

% Main title of the paper
\title [mode = title]{Formal verification of TLS with CiMPG}  

% Title footnote mark
% eg: \tnotemark[1]
\tnotemark[1] 

% Title footnote 1.
% eg: \tnotetext[1]{Title footnote text}
\tnotetext[<tnote number>]{<tnote text>} 

% First author
%
% Options: Use if required
% eg: \author[1,3]{Author Name}[type=editor,
%       style=chinese,
%       auid=000,
%       bioid=1,
%       prefix=Sir,
%       orcid=0000-0000-0000-0000,
%       facebook=<facebook id>,
%       twitter=<twitter id>,
%       linkedin=<linkedin id>,
%       gplus=<gplus id>]

% \author[<aff no>]{<author name>}[<options>]
\author{}
% Corresponding author indication
% \cormark[<corr mark no>]

% Footnote of the first author
% \fnmark[<footnote mark no>]

% Email id of the first author
\ead{<email address>}

% URL of the first author
\ead[url]{<URL>}

% Credit authorship
% eg: \credit{Conceptualization of this study, Methodology, Software}
\credit{<Credit authorship details>}

% Address/affiliation
\affiliation[<aff no>]{organization={},
            addressline={}, 
            city={},
%          citysep={}, % Uncomment if no comma needed between city and postcode
            postcode={}, 
            state={},
            country={}}

% \author[<aff no>]{<author name>}[<options>]

% Footnote of the second author
% \fnmark[2]

% Email id of the second author
\ead{}

% URL of the second author
\ead[url]{}

% Credit authorship
\credit{}

% Address/affiliation
\affiliation[<aff no>]{organization={},
            addressline={}, 
            city={},
%          citysep={}, % Uncomment if no comma needed between city and postcode
            postcode={}, 
            state={},
            country={}}

% Corresponding author text
\cortext[1]{Corresponding author}

% Footnote text
\fntext[1]{}

% For a title note without a number/mark
%\nonumnote{}

% Here goes the abstract
\begin{abstract}
TLS protocol has been formally verified with CiMPG in CafeInMaude. The formal specification of TLS is modeled as transition systems in terms of equations in CafeOBJ/OTS method, and it has been verified that the system have properties by means of equational reasoning. The given initial proof scores in CafeOBJ are revised and added some additional proof fragments in some invariants if it is necessary such that the proof scores are complete and they could be successfully tackled by CiMPG. When the complete and correct proof scores are available, they are verified in CiMPG. Feeding the proof scores into CiMPG, CiMPG can generate proof scripts for CiMPA. However, it is not a good idea to feed all proof scores of all invariants in CiMPG as one file because CiMPG takes too much time to generate the proof scripts since the size of the proof score is quite large. Thus, each invariant has been tackled with CiMPG one by one.
\end{abstract}

% Use if graphical abstract is present
%\begin{graphicalabstract}
%\includegraphics{}
%\end{graphicalabstract}

% Research highlights
% \begin{highlights}
% \item 
% \item 
% \item 
% \end{highlights}

% Keywords
% Each keyword is seperated by \sep
\begin{keywords}
 CiMPG \ CafeInMaude \ invariant \ OTS 
\end{keywords}

\maketitle

% Main text
\section{Introduction}\label{intro}
Internet security has become the important topic in recent years. The rapid development of e-commerce, individuals and organizations are relying on the Web and security services such as online banking, online shops etc, that need to exchange the sensitive information such as credit card numbers, account numbers, etc. between two parties in non-secure network environment. With the increase service availability the number of risks increase both for users and providers. In order to ensure a secure communication between them, a large number of security protocols have been designed and developed to protect the data. Among them, Transport Layer Security protocol (TLS)\cite{dierk} is the most widely used security protocol. TLS is the successor of Secure Sockets Layer (SSL)\cite{1621007}. 
% here

TLS was mainly developed in order to give the customers
confidence in their interactions when they access online services
from their electronic devices. If two parties want to securely establish a common key over an insecure channel, they typically execute a key exchange protocol. Both TLS and SSL protocols confine a set of rules for the communication between client and server. They rely on public-key cryptography in order to ensure integrity of exchanged data. Although  these  authentication  protocols  have been carefully designed by security experts, it is not uncommon for security attacks  as  technology  continues  to  advance. Despite multiple prevention measurements, several vulnerabilities such as  
Heartbleed\cite{10.1145/2663716.2663755} and DROWN\cite{197245} have been discovered. Therefore, ensuring the reliability of security protocols is really important and more effort is required to test the implementations of such security protocols.

One promising technology is formal verification  with theorem  proving. Theorem proving has been used to prove that systems enjoy some desired properties by writing proof scores. Proof scores are written as programs in an algebraic specification language to conduct formal verification. This approach uses observational transition systems (OTSs) \cite{OgataF03fmoods} as state machines to formalize systems. A security protocol,together with the most generic intruder, is modeled as an OTS. Then, the OTSs are specified in equations with CafeOBJ\cite{DiaconescuF98amast}, a formal specification language. System properties are expressed as CafeOBJ terms. Formal verification is conducted by writing what is called ``proof scores'' \cite{OgataF03fmoods} in CafeOBJ and executing them with CafeOBJ. This approach to formal verification is flexible, however, the proof may contain some flaws since proof scores are subject to human errors. 

CafeInMaude is the second implementation in Maude of CafeOBJ, where Maude~\cite{ClavelEtal2007maude} is a sibling language of CafeOBJ.
CafeInMaude introduces CafeOBJ specifications into the Maude system.
It comes with two extension tools CafeInMaude Proof Assistant (CiMPA) and CafeInMaude Proof Generator (CiMPG)~\cite{RiescoO18tosem}. CiMPA is a proof assistant that allows users to write proof scripts in order to prove invariant properties on their CafeOBJ specifications. Using CiMPA to develop the formal verification by writing proof scripts can help us to avoid the flaw made by human users. However, it is often the case that CiMPA is not flexible enough to conduct formal verification, especially when the size of proof score is huge. 

CiMPG allows users to combine the flexibility of the proof score approach and the reliability of CiMPA by providing a minimal set of annotations for identifying proof scores and generating CiMPA scripts for these proof scores. Given slightly annotated proof scores, CiMPG generates proof scripts for CiMPA. Feeding the generated proof scripts into CiMPA, if CiMPA successfully discharges all goals, the proof scores are correct for the goals and the formal verification is successful. Although CiMPG can automatically generate proof scripts for CiMPA from annotated proof scores, it will take time to do so when the size of the input proof score is large.

We have formally verified academic protocols
such as NSLPK authentication protocol \cite{twmon}. TLS protocol is much more complicated than academic protocols. Although it has been formally analyzed and verified that TLS enjoys some desired properties in \cite{1437139} with OTS/CafeOBJ method, we are the first to formally verify that TLS enjoys some desired properties with CiMPG as  well as CiMPA. 
All specifications and proofs presented in this paper are available at \url{https://github.com/twmon14/fvtls}.

The rest of the paper is organized as follows. Section \ref{tls} mentions TLS, TLS handshake protocol and the abstract handshake protocol. Section \ref{fstls} presents the formal specification of TLS. Section \ref{fvtls} describes the formal verification of TLS with CiMPG. Some related work is mentioned in Section \ref{relatedwork}. Finally, we give our conclusions and some lines of future work in Section \ref{conclusion}.
	
\section{TLS}\label{tls}
TLS\cite{dierk} provides privacy and data integrity between client/server communications. The protocol consists of four sub-protocols. They are TLS handshake protocol, the change cipher spec protocol, the alert protocol and TLS record protocol. 

When two peers want to establish a communication, TLS handshake protocol is used to negotiate a cipher suite and security parameters between client and server. The change cipher spec protocol is used when each of two peers(client/server) want to notify the other that subsequent messages after handshake process will be protected under the newly negotiated cipher suite,  and security parameters. If either of them notices something wrong, one lets the other know about it with the alert protocol. TLS record protocol provides private and reliable communications between the two peers using the negotiated cipher suite and security parameters.

If a cipher suite used is weaker than the available one to both peers or the security parameters shared by both peers are leaked by some reasons, then an adequate level of security cannot be obtained. Therefore, the security of TLS largely depends on a cipher suite and security parameter negotiated by client/ server in handshake process. It can be said that among four sub-protocols, TLS handshake protocol takes part in the most important role and the security of TLS relies on TLS handshake protocol. 

\subsection{TLS handshake protocol}\label{handshake}
Figure 1 shows the message exchanges between client and server in TLS handshake protocol. 

To initiate a new run of the protocol, a server may send a HelloRequest message to a client or a client sends a ClientHello message to the server. On receipt of the HelloRequest message from server, the client replies to server with a ClientHello message. A ClientHello message contains the basic information such as a version number of TLS, a list of cipher suites that are available to the client, a random number generated by client, etc.

The server then replies to the ClientHello message with a
ServerHello message, which consists of a version number of TLS that is lower than the one suggested by the client and the highest supported by the server, a session ID, a cipher suite chosen from the list suggested by the client in the ClientHello message, a random number generated by server, etc. 

If the server should be authenticated, the server sends  his/her certificate to the client. If the server Certificate message does not contain enough data to allow the client to exchange a parameter secret called a pre-master secret, the server sends a ServerKeyExchange message to the client. The server can optionally request a certificate from the client by sending a CertificateRequest message. The server sends a ServerHelloDone message to the client to indicate the end of the ServerHello.

When the client receives ServerHelloDone message, the client checks the messages received. When it is necessary, the client sends his/her certificate to the server. If it is sent, then a ClientKeyExchange message is always sent by the client after the client Certificate message. Otherwise, the ClientKeyExchange message is the first message sent by the client after he/she receives a ServerHelloDone message. The ClientKeyExchange message
contains a pre-master secret which is a random number generated by the client. Two key exchange methods (RSA and Diffie-Hellman) are specified to exchange a pre-master secret. 

After ClientKeyExchange message, the client can optionally send a CertificateVerify message to explicitly provide a verification of the client certificate.
The client notifies the server with the change cipher spec protocol that subsequent messages will be protected under the newly negotiated cipher suite and security parameters. The client then sends a Finished message to the server to verify that the handshake process
is finished successfully. The Finished message is protected under the just negotiated cipher suite and security parameters.

On receipt of the client Finished message, the server
checks the message received. If the message is correct, the server notifies the client with the change cipher spec protocol that subsequent messages will be protected under the newly negotiated cipher suite and security parameters. The server then sends a server Finished message to the client to verify that the handshake process is successful.

On receipt of the server Finished message, the client
checks the message received. If the message is correct, the handshake process is done successfully between client and server and the client can exchange application data securely with server in the current session.

A previously established session and a current session
can be resumed and/or duplicated without exchanging all handshake messages. If a client wants to resume or duplicate a session, he/she sets the session ID in a new ClientHello message and sends it to a server. The ClientHello message is protected under the current cipher suite and security parameters. On receipt of the ClientHello message, if the server wants to resume or duplicate the session, he/she replies to the client with a new ServerHello message. The ServerHello message is protected under the current cipher suite and security parameters. 

The server then signals the client to use the newly negotiated security parameters for the session. The same cipher suite is used. A server Finished message is then sent to the client by the server. On receipt of the server Finished message, the client checks the message received. If the message is correct, the client signals the server to use the newly negotiated security parameters for the session and a client Finished message is then sent to the server. On receipt of the client Finished message, the server checks the message. If the message is correct, the session is then resumed and/or duplicated and the client can exchange application data securely with server in that session.

\subsection{The Abstract Handshake Protocol}\label{abstracthandshake}
Figure 2 shows the abstract version of TLS handshake protocol. In the protocol, A denotes a client and B a server respectively. In the figure, the first six exchanged messages are the full handshake process in which a cipher suite and security parameters are negotiated between client and server. The remaining ones are for the resumption of a previously established session or the duplication of a current session.

Cryptographic functions used in the protocol are H(.)
(a one-way hash function), $\varepsilon_K(.)$ (an encryption function with symmetric or asymmetric key $K$) and $S_X(.)$(a digital signature function with principal $X$’s private key). The other primitives used in the protocol are Rand$_X$ (a random number generated by principal $X$), ListOfChoices (a list of cipher suites), Choice (a cipher suite), SID (a session ID), PMS (a pre-master secret), CA (certificate authority) and K$_X$ (principal $X$'s public key).

Composite data occurring in the protocol are as follows:
Cert$_X$ : $X$, K$_X$, $S_{CA}$($X$, K$_X$) \newline
ClientKey : H(A, PMS, Rand$_A$, Rand$_B$) \newline
ClientKey : H(B, PMS, Rand$_A$, Rand$_B$) \newline
ClientFinish : H("client", $A$, $B$, SID, ListOfChoices, Choice, Rand$_A$, Rand$_B$, PMS)\newline
ServerFinish : H("server", $A$, $B$, SID, ListOfChoices, Choice, Rand$_A$, Rand$_B$, PMS)\newline
ClientFinish2 : H("client", $A$, $B$, SID, Choice, Rand$_A$, Rand$_B$, PMS)\newline
ServerFinish2 : H("server", $A$, $B$, SID, Choice, Rand$_A$, Rand$_B$, PMS)\newline

We suppose the following in the abstract handshake protocol: a server always sends his/her certificate to a client when a full handshake process is performed; a server sends neither SeverKeyExchange nor CertificateRequest messages; server Certificate message also plays the role of ServerHelloDone messages; a client sends neither Certificate nor CertificateVerify messages; ChangeCipherSpec messages are implicit; there exists only one trusted certificate authority; the method used for exchanging pre-master secrets is RSA only; the content of Finished messages is not the hash of the security parameters and handshake messages exchanged, but that of two random numbers and a pre-master secret, etc.
% 		\end{verbatim}
% 	\end{small}

\section{Formal Specification of TLS} \label{fstls}
Suppose that there exist not only multiple trustable
principals but also multiple malicious (untrustable) principals, and the cryptosystem used is perfect. Trustable
principals exactly follow the protocol, while malicious ones may do something against the protocol as well. The combination and cooperation of malicious principals is modeled as the most general intruder Dolev-Yao\cite{dolev}. 
An intruder can glean as much information as possible from messages flowing in the network and create fake messages based on the gleaned information, provided that the intruder cannot break the perfect cryptosystem.

We specify 10 operators (data construtors) to define 10 kinds of messages. The data constructors used in TLS are declared as follows:
\begin{small}
		\begin{verbatim}
  op ch  : Prin    Prin   Prin  Rand ListOfChoices -> Msg
  op sh  : Prin    Prin   Prin  Rand Sid Choice    -> Msg
  op ct  : Prin    Prin   Prin  Cert               -> Msg
  op kx  : Prin    Prin   Prin  EncPms             -> Msg
  op cf  : Prin    Prin   Prin  EncCFin            -> Msg
  op sf  : Prin    Prin   Prin  EncSFin            -> Msg
  op ch2 : Prin    Prin   Prin  Rand Sid           -> Msg
  op sh2 : Prin    Prin   Prin  Rand Sid Choice    -> Msg
  op cf2 : Prin    Prin   Prin  EncCFin2           -> Msg
  op sf2 : Prin    Prin   Prin  EncSFin2           -> Msg
		\end{verbatim}
	\end{small}
where \verb!Msg! is the sort denoting messages. 

The other sorts and the corresponding data
constructors that constitute the messages are declared as follows:
\begin{itemize}
  \item \verb!Principal! denotes principals. There are two special principals; one is the \verb!intruder! denoted by intruder and the other the certificate authority denoted by \verb!ca!. We suppose that \verb!intruder! does not equal \verb!ca!. 
  \item \verb!Rand! denotes random numbers generated by principals.
\item \verb!ListOfChoices! denotes lists of cipher suites.
  \item \verb!Choice! denotes cipher suites. 
  \item \verb!Sid! denotes session IDs. 
  \item \verb!PubKey! denotes public keys. The principal $a$’s public key is denoted by \verb!k!($a$).
\item \verb!Sig! denotes digital signatures of pairs of a principal and a public key. The digital signature of the pair of principal $b$ and public key $k$ signed by $ca$ is denoted by \verb!sig($ca$, $b$, $k$)!.
\item \verb!Cert! denotes certificates of public keys. Given principal $a$, public key $k$ and signature $g$, the certificate that $k$ is $a$’s,
which is certified by $g$, is denoted by \verb!cert!($a$, $k$, $g$).
\item \verb!Secret! denotes secret values that make pre-master secret globally unique and unguessable. 
\item \verb!Pms! denotes premaster secrets. Given two principals $a,b$ and a secret value $s$, a pre-master secret generated by client $a$ for server $b$ is denoted by \verb!pms!($a$, $b$, $s$).
\end{itemize}

Operator \verb|_\in_| is the membership predicate of lists.

For data constructor \verb!x! (where \verb!x = ch!, \verb!sh!, \verb!ct!, \verb!kx!, \verb!cf!, \verb!sf!, \verb!ch2,! \verb!sh2!, \verb!sf2!, \verb!cf2!), predicate \verb!x!? is defined, which checks if a given message is \verb!x! message. Given a term denoting a message, projection operators such as \verb!crt!, \verb!src! and \verb!dst! return the first, second and third arguments of the term, respectively.The first, second and  third  arguments  of  each  message are  the  actual creator,  the  seeming  sender  and  the  receiver  of the  corresponding message. The first argument is meta-information that is only available to the outside observer and the principal that has sent the corresponding message, and that cannot be forged by the intruder; while the remaining arguments may be forged by the intruder. The projection operators \verb!rand, list, choice, sid, cert, epms, ecfin, esfin, ecfin2! and \verb!esfin2! return other arguments, respectively.

Suppose that there exists a message in the network. It is true that the principal denoted by the first
argument has sent the message. If the first argument is the
intruder and the second one is not, then the message has been
faked by the intruder. For example, a principal $a$ receives a message denoted by \verb!ct!($b$',$b$,$a$,$cert$). 
Since messages are supposed to be never deleted from the network, if there exists a message denoted by
\verb!ct!($b$,$b$,$a$,$cert$) the network, we can conclude that the message received by $a$ really originates from $b$.
Therefore, this formalization of messages makes it possible to describe a property such as one that a message received by a principal really originates from the seeming sender (second argument) of the message. 



\section{Formal Verification of TLS with CiMPG}\label{fvtls}



\section{Related Work}\label{relatedwork}
% The paper ~\cite{spin} used the model checking method to formally verify security protocols using Spin model checker. The paper exampled NSPK protocol and DS protocol. The experimental results showed that
% the number of the protocol model states has been decreased
% by a wide margin, and the efficiency of protocol verification has been improved properly. The security protocol models mainly include the protocol
% instance model and the intruder model. It is described with Promela Model. The property of security protocols is described using LTL. An attack is found in each protocol. The difference between their method and our method is that we use OTS/CafeOBJ method to formalize the protocols as state machines. The Dolev-Yao intruder model is used in both methods. Furthermore, this paper only verified the authentication property of security protocols while we verified authentication and secrecy properties.

% Mitchell, et al.[7] use the model checker Mur [3] to
% check seven simple protocols derived from the SSL 3.0
% handshake protocol. The primal reason why they have analyzed
% the protocols is to identify the purpose of certain message
% fields (version number, nonce, etc.) in some steps of
% the protocol. The analysis starts with the simplest version of
% the handshake protocol from which some fields are omitted,
% and the fields are gradually added to the protocol. The first
% six protocols have been found badly flawed and the model
% checker has found many attacks. The model checker has
% been used to check the final protocol with two clients, one
% server, no more than two simultaneous open sessions per
% server and no more than one resumption per session, and no
% attacks have been discovered.

% Paulson[11] analyzes the TLS handshake protocol with
% his inductive method[10] that is supported by the proof assistant
% Isabelle/HOL[8]. In the protocol analyzed by Paulson,
% servers always sent their certificates to clients, the key exchange method considered is RSA, and clients optionally
% send their certificates and ClientKeyExchange messages to
% servers. In his model of the protocol, a malicious principal
% called the spy, which corresponds to the intruder in our
% model, is taken into account and it is supposed that any session
% key, if used, may end up in the hands of the spy, which
% is denoted by the rule Oops. One of the results of the analysis
% is that session resumption turns out to be safe even if the
% spy has obtained session keys from earlier sessions.

% Palombo, et al. \cite{Palombo2015ACS} have formally analyzed several examples of a particular class of cryptographic protocols (NSPK, DS, TMN, and DH) using three alternative tools. They used SPIN to analyze explicit state models written in Promela; symbolic models in the pi-calculus were verified using Proverif, and finally, we proved lemmas that show protocol insecurity using the theorem prover Coq. They have presented that choosing the right tool always implies a trade-off. It is generally easier to model a system using SPIN’s state representation than using the pi-calculus or
% logical inference rules. However, SPIN can only verify a bounded number of sessions and
% message space. Proverif verifies an unbounded number of sessions and message space at the cost of potentially not terminating or reporting false attacks. On the other hand, Coq theories with security proofs are not bounded and can
% represent an unrestricted symbolic attacker. They have shown that protocol insecurity can be
% proved in Coq. On the other hand, proof that demonstrates that the properties hold may be harder to construct. In comparison with our research, we proved lemmas that show protocol security using proof assistant CiMPA and proof generator CiMPG.

\section{Conclusion}\label{conclusion}

% Numbered list
% Use the style of numbering in square brackets.
% If nothing is used, default style will be taken.
%\begin{enumerate}[a)]
%\item 
%\item 
%\item 
%\end{enumerate}  

% Unnumbered list
%\begin{itemize}
%\item 
%\item 
%\item 
%\end{itemize}  

% Description list
%\begin{description}
%\item[]
%\item[] 
%\item[] 
%\end{description}  

% Figure
% \begin{figure}[<options>]
% 	\centering
% 		\includegraphics[<options>]{}
% 	  \caption{}\label{fig1}
% \end{figure}


% \begin{table}[<options>]
% \caption{}\label{tbl1}
% \begin{tabular*}{\tblwidth}{@{}LL@{}}
% \toprule
%   &  \\ % Table header row
% \midrule
%  & \\
%  & \\
%  & \\
%  & \\
% \bottomrule
% \end{tabular*}
% \end{table}

% Uncomment and use as the case may be
%\begin{theorem} 
%\end{theorem}

% Uncomment and use as the case may be
%\begin{lemma} 
%\end{lemma}

%% The Appendices part is started with the command \appendix;
%% appendix sections are then done as normal sections
%% \appendix

% There are several ideas/devices that should be used to make the formal
% verification successful.


% To print the credit authorship contribution details
\printcredits

%% Loading bibliography style file
%\bibliographystyle{model1-num-names}
\bibliographystyle{cas-model2-names}

% Loading bibliography database
\bibliography{cas-refs}

% Biography
\bio{}
% Here goes the biography details.
\endbio

% \bio{pic1}
% % Here goes the biography details.
% \endbio

\end{document}

% 